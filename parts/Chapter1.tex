\section{Постановка задачи}
\label{sec:Chapter1} \index{Chapter1}

Главной задачей является улучшить генерации модели \textsc{CADRecode}, обученной на синтетических данных. Этого я собираюсь добиться путем изменения train данных,
а именно формата CAD представления и самого распределения 3D объектов для обучения.
Именно поэтому я ставлю перед собой следующий задачи:
\begin{enumerate}
	\item Определить какой формат представления CAD объекта лучше
	\item На предпочтительном формате внедрить изменения в train данные, которые повысят метрики эксперимента
\end{enumerate}

Таким образом задачи поставлены так, чтобы понять каким образом исходные данные влияют на работу модели и в итоге улучшить ее работу.
В данном случае первое - это различные форматы представления CAD, второе - различные синтетические датасеты.

\newpage
