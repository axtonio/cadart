\section{Введение}
\label{sec:Chapter0} \index{Chapter0}

В данной работе рассматривается задача генерации CAD-объектов из облаков точек. Хотя входными данными для модели могут служить как текстовое описание инженерной задачи, так и изображения или наброски, наибольший интерес представляет именно \textit{point cloud}, поскольку в промышленных отраслях всё чаще применяется 3D-сканирование и обратная разработка (reverse engineering), позволяющие воссоздавать CAD-модели из облаков точек.

Важно уточнить, что CAD "--- это не просто 3D-модель: она задаётся последовательностью команд (constructive sequence), которые, в свою очередь, реализуются геометрическими движками. Набор команд в разных движках может отличаться, однако базовый список обычно включает:
\begin{enumerate}
	\item Sketch -- команда, позволяющая создавать 2D-эскизы, задавая контуры и элементы,
	      пригодные для дальнейшего 3D-моделирования
	\item Extrude -- выдавливает (экструзирует) выбранный эскиз, превращая 2D-контур в
	      трёхмерную деталь заданной высоты
	\item Fillet -- сглаживает или скругляет выбранные ребра для создания плавных
	      переходов между поверхностями
	\item Chamfer -- срезает выбранные ребра под заданным углом и на заданное расстояние,
	      формируя фаску
	\item Mirror -- копирует выбранную геометрию зеркально относительно заданной
	      плоскости или плоскости симметрии
	\item CircularPattern -- создает массив (паттерн) из выбранных элементов, равномерно
	      расположенных по окружности
	\item Shell -- превращает твердое тело в полую оболочку, задавая толщину стенок и
	      удаляя выбранную грань
	\item Sweep -- выдавливает (протягивает) профиль вдоль заданного пути (траектории),
	      создавая более сложные объёмные формы
	\item Revolve -- вращает 2D-профиль вокруг оси, формируя осесимметричное твердое тело
\end{enumerate}

Поставленная задача имеет множество ограничений:
\begin{enumerate}
	\item \textbf{Отсутствие унифицированного формата представления CAD-объектов.}
	      Существует множество форматов представления CAD: зарубежные движки (Fusion360, Autodesk, Onshape) и российские (например, КОМПАС-3D) имеют собственные реализации и кодовые представления, которые зачастую не совместимы друг с другом.

	\item \textbf{Нехватка данных и бенчмарков.}
	      Несмотря на то, что инженерные модели повсеместно встречаются в различных отраслях, компании не публикуют в открытый доступ соответствующие датасеты, опасаясь утечки технологий и финансовых потерь.
	      На сегодняшний день крупнейший датасет ABC содержит около 1 млн реальных образцов, чего недостаточно для обучения полноценной обобщающей модели.
	      Кроме того, в нём невозможно извлечь \textit{constructive sequence}, на которых базируется всё обучение.
	      По сути, наибольшим (но всё ещё ограниченным) датасетом реальных данных является DeepCAD размером в 270\,тыс.\ образцов, включающий лишь операции \textit{sketch-extrude}.
	      Также отсутствуют бенчмарки, позволяющие проверить умения системы по отдельным сложным инженерным навыкам, например, по построению сложного скетча.

	\item \textbf{Отсутствие конструктивных метрик.}
	      Наиболее распространёнными метриками качества считаются \textit{chamfer distance} (расстояние между облаками точек) и \textit{intersection over union}, но они не отражают способность модели к более «инженерному» обобщению.
	      Например, сложную модель можно с достаточной точностью аппроксимировать набором простых примитивов, что удовлетворит вышеуказанные метрики, однако это не продемонстрирует реальную способность модели использовать предусмотренные операции CAD.

	\item \textbf{Слабые бейзлайны.}
	      В настоящее время все модели для генерации CAD обучены лишь на датасетах, поддерживающих ограниченный набор операций (в основном \textit{sketch-extrude}), и на промышленных данных показывают непригодные для реального применения результаты.
	      По сути, не существует ни универсальной архитектуры, ни метода обучения, которые бы существенно превосходили существующие решения и позволяли сравнивать их напрямую, поскольку многие из существующих пайплайнов специфичны и мало сопоставимы друг с другом.
\end{enumerate}

Перечисленные ограничения наглядно демонстрируют, что задача генерации CAD-объектов по облаку точек пока не решена и вряд ли будет закрыта в ближайшем будущем.
Требуется разработка новых датасетов, бенчмарков и правил, подкреплённых высокоинформативными метриками.
В настоящей работе будут затронуты все четыре проблемы

Приведём список используемых в работе понятий с их определениями:
\begin{description}
	\item[\textbf{CAD}] %
	      Компьютерное проектирование (Computer-Aided Design), создание 2D/3D-моделей.
	\item[\textbf{Селектор}] %
	      Инструмент или функция, позволяющая выбирать (выделять) объекты, элементы или параметры в модели или интерфейсе программы.
	\item[\textbf{Квантизация}] %
	      Процесс уменьшения точности числовых данных (например, весов нейронной сети) для оптимизации вычислений.
	\item[\textbf{B-rep (Boundary Representation)}] %
	      Граничное представление, описание 3D-объекта через грани, рёбра и вершины.
	\item[\textbf{Constructive sequence (CS)}] %
	      Конструктивная последовательность, история операций при моделировании в CAD.
	\item[\textbf{CSG (Constructive Solid Geometry)}] %
	      Конструктивная блочная геометрия, построение сложных тел через булевы операции.
	\item[\textbf{Mesh}] %
	      Полигональная сетка, представление 3D-модели в виде вершин, рёбер и граней.
	\item[\textbf{Point cloud}] %
	      Облако точек, массив 3D-координат, описывающий форму объекта.
	\item[\textbf{Onshape}] %
	      Облачная CAD-система для совместной работы.
	\item[\textbf{Synthetic data}] %
	      Синтетические данные, искусственно сгенерированные для обучения ИИ.
	\item[\textbf{Reinforcement learning}] %
	      Обучение с подкреплением, метод ИИ, основанный на наградах и штрафах.
	\item[\textbf{Finetuning}] %
	      Дообучение модели под конкретную задачу с настройкой параметров.
	\item[\textbf{Pretrain}] %
	      Предобучение, начальная тренировка модели перед дообучением.
	\item[\textbf{Sample}] %
	      Образец данных (например, одно изображение в датасете).
	\item[\textbf{STEP (Standard for the Exchange of Product Data)}] %
	      Стандартный формат обмена CAD-данными (ISO~10303).
	\item[\textbf{STL (Standard Tessellation Language)}] %
	      Стандартный формат представления полигональных сеток.
	\item[\textbf{JSON (JavaScript Object Notation)}] %
	      Текстовый формат обмена данными, основанный на структурах <<ключ--значение>>.
	\item[\textbf{Python}] %
	      Высокоуровневый язык программирования общего назначения.
	\item[\textbf{CadQuery}] %
	      Python-библиотека для параметрического 3D-моделирования на базе OpenCASCADE.
	\item[\textbf{OpenCASCADE}] %
	      Открытая CAD-библиотека ядра для 3D-моделирования и обработки геометрии.
	\item[\textbf{PythonOCC}] %
	      Python-обёртка для OpenCASCADE, позволяющая использовать его функционал в Python.
\end{description}

\newpage
