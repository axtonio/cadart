\begin{abstract}

    \begin{center}
        \large{Разработка системы автоматизированного проектирования PointCloud2CAD} \\
        \large\textit{} \\[1 cm]
    \end{center}
    % Краткое описание задачи и основных результатов, мотивирующее прочитать весь текст.

    \noindent Система автоматизированного проектирования (CAD) широко применяется в инженерной отрасли,
    однако реализация промышленных проектов в формате «end-to-end» зачастую оказывается крайне затратной и продолжительной по времени.
    Например, создание 3D-модели автомобиля методом реверс-инжиниринга из облака точек может растянуться на три месяца и потребовать вложений порядка полумиллиона.
    Дополнительным препятствием является нехватка в России квалифицированных специалистов по 3D-инжинирингу.
    Все эти факторы подчеркивают высокую актуальность задачи генерации CAD-объектов, которая уже привлекает внимание многих компаний.
    Однако на сегодняшний день эта задача не решена и остается сложной из-за дефицита крупных наборов данных и отсутствия унифицированного формата представления.

    В данной работе поставлена цель улучшить генерации модели \textsc{CADRecode}, обученной на синтетических данных.
    Для этого была выполнена модернизация генератора синтетических данных путём добавления пользовательских скетчей,
    а также проведён сравнительный анализ двух форматов представления CAD-объектов на специально собранном валидационном наборе.
    Этот набор позволяет оценить способность обученной модели корректно воспроизводить инженерные эскизы.

    Итоги экспериментов показывают, что предложенный подход положительное влияние на работу модели. Улучшая основные метрики CD и IoU и визуальное качество генераций
    \vfill
    \begin{center}
        \textbf{Abstract} \\[1 cm]

        Разработка системы автоматизированного проектирования PointCloud2CAD
    \end{center}

\end{abstract}
\newpage