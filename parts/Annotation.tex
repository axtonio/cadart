\begin{abstract}

    \begin{center}
        \large{Разработка системы автоматизированного проектирования PointCloud2CAD} \\
        \large\textit{} \\[1 cm]
    \end{center}
    % Краткое описание задачи и основных результатов, мотивирующее прочитать весь текст.

    \noindent CAD (Computer-Aided Design) широко применяется в инженерном проектировании. Однако современная «end-to-end» реализация промышленных проектов обходится дорого и требует значительных временных затрат. Например, создание 3D-модели автомобиля может занять до трёх месяцев и стоить около полумиллиона. Кроме того, на российском рынке ощущается нехватка квалифицированных специалистов в сфере 3D-инжиниринга. Все эти факторы указывают на высокую актуальность задачи генерации CAD-объектов, вызывающей интерес у многих компаний. На сегодняшний день эта задача не решена и остаётся сложной из-за нехватки больших наборов данных и отсутствия унифицированного формата представления.

    В работе представлен новый синтетический датасет, позволяющий улучшить метрики state-of-the-art-решения \textsc{CADRecode} (без учёта RL-дообучения).
    Я сравнил два кодовых представления CAD-моделей на специально собранном валидационном наборе, оценивающем способность системы воспроизводить инженерные эскизы, и выбрал более эффективное.
    Также в статье описана модернизация генератора синтетических данных CADRecode путём интеграции скетчей, созданными людьми.

    \vfill
    \begin{center}
        \textbf{Abstract} \\[1 cm]

        Разработка системы автоматизированного проектирования PointCloud2CAD
    \end{center}

\end{abstract}
\newpage